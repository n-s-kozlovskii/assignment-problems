\documentclass[10pt,a4paper]{article}
\usepackage[utf8]{inputenc}
\usepackage[russian]{babel}
\usepackage[OT1]{fontenc}
\usepackage{amsmath, amsfonts, amssymb}
\author{Козловский Никита}
\title{Трехиндексные аксиальные задачи о назначениях}
\begin{document}
\section{abstract}
Задачи о назначениях часто встречаются в повседневной жизни: составление расписаний,
распределение обязаностей между работниками и многие другие. Актуальность задачи
привела к тому, что для простой, двухиндексной постановки были получены эффективные
алгоритмы решения. Однако при расширении задачи до трехиндексной было установлено, что
такая задача отностится к классу $\mathcal{NP}$ полных, поэтому для неё существует множество
алгоритмов. В этой работе мы рассмотрим некоторые из них и подробно исследуем алгоритм, предложенный
Э. Х. Гимади, который состоит в уменьшении размерности решаемой задачи.

\section{Основные понятия}
Для начала дадим классическое определение задачи о назначениях.
Введем понятие назначения. Мы можем
представлять назначение как некое биективное отображение $\phi$, которое ставит
элементы конечного множества $\mathrm{U}$ в соотвествие элементам конечного
множества $\mathrm{V}$. В тоже время назначение является перестановкой, которая записывается
в виде

\[
\left (
  \begin{tabular}{cccc}
  1 & 2 & \ldots & n\\
  $\varphi (1)$ & $\varphi (2)$ & \ldots & $\varphi (n)$
  \end{tabular}
\right )
\]

Каждой перестановке множества $\{1, 2, \ldots , n \}$ соответсвует единственная матрица
перестановок $\mathrm{X}_\varphi \in \mathrm{Matrix}_{n \times n}$, элементы котороый определяются как
\[
x_{ij} =
 \begin{cases}
   1 & \text{если } j = \varphi(i) \\
   0 & \text{иначе}
 \end{cases}
\]

Обозначим множество $\mathrm{S}_n$ как множество всех возможных перестановок множества
 $\{1, 2, \ldots , n \}$. Мощность этого множества $n!$.

\subsection{Линейная задача о назначениях}
Пусть дана матрица $n \times n$ весовых коэфициентов $C = (c_{ij})$.
Требуется минимизировать линейную форму

\[
  \sum^n_{i = 1} c_{i \varphi (i)}
\]

то есть линейная задача о назначениях может быть поставлена в виде

\[
  \min_{\varphi \in S_n} \sum^n_{i = 1} c_{i \varphi (i)}
\].

При этом, если перестановки задаются матрицей перестановок $\mathrm{X} = (x_{ij})$,
линейная задача о назначениях может быть записана как задача линейной оптимизации
\begin{eqnarray*}
  & \min \displaystyle \sum^n_{i = 1} \displaystyle \sum^n_{j = 1} c_{ij} x_{ij} \\
  \text{s.t.} & \displaystyle \sum^n_{i = 1} x_{ij} = 1 &(j = 1, \ldots, n) \\
  &\displaystyle \sum^n_{j = 1} x_{ij} = 1 &(i = 1, \ldots, n) \\
  & x_{ij} \in \{ 0, 1 \}
  &(i,j = 1, \ldots, n)
\end{eqnarray*}

\textit{Пример}

Пусть требуется назначить $n$ работников на $n$ работ наиболее эфективным образом.
Предположим, что $j$ работнику требуется $c_ij$ времени, чтобы выполнить работу
$i$. Тогда нужно оптимизировать функцию
$\sum^n_{i = 1} c_{i \varphi(i)} $

\section{Трехиндексная задача о назначениях}

Рассмотрим следующую задачу.
Составим расписание занятий, то есть нужно распределим $n$ занятий
среди $n$ свободных временных промежутков и среди $n$ свободных аудиторий.
Пусть $c_{ijk}$ соответсвует "стоимости" назначения курса $i$ на время $j$
в аудиторию $k$. Мы хотим найти такую перестановку $\varphi$, которая ставит
в соответствие курс к временному промежутку и такую перестановку $\psi$,
которая ставит
в соответствие курс к свободной аудитории, что "стоимость" назначения будет
минимальна. Эта задача -- пример трехиндексной аксиальной задачи о назначениях.
\[
 \min_{\varphi , \psi \in S_n} \sum^{n}_{i = 1} c_{i \varphi (i) \psi (i) }
\]

Также можно переписать эту задачу в виде

\begin{eqnarray*}
  & \min \displaystyle \sum^n_{i = 1} \displaystyle \sum^n_{j = 1} \displaystyle \sum^n_{k = 1}
  c_{ijk} x_{ijk} \\
  \text{s.t.}
  &\displaystyle \sum^n_{i = 1} \displaystyle \sum^n_{k = 1} x_{ijk} = 1  &(j = 1, \ldots, n) \\
  &\displaystyle \sum^n_{j = 1} \displaystyle \sum^n_{k = 1} x_{ijk} = 1  &(i = 1, \ldots, n) \\
  &\displaystyle \sum^n_{i = 1} \displaystyle \sum^n_{j = 1} x_{ijk} = 1  &(k = 1, \ldots, n) \\
  & x_{ijk} \in \{ 0, 1 \} &(i,j,k = 1, \ldots, n)
\end{eqnarray*}

Р. М. Карп показал, что эта задача является $\mathcal{NP}$ полной. 
\end{document}
