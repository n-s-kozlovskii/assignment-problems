\documentclass[14pt,a4paper]{article}
\usepackage[utf8]{inputenc}
\usepackage[russian]{babel}
\usepackage[OT1]{fontenc}
\usepackage{amsmath, amsfonts, amssymb}
\usepackage{listings}
\usepackage[small]{titlesec}
\author{Козловский Никита}
\title{Трехиндексные аксиальные задачи о назначениях}
\begin{document}

\section{abstract}
Задача о назначениях в различных постановках часто встречаются в повседневной
жизни. В учебных заведениях возникает потребность составить грамотное расписание,
для каждого работника на предприятии необходимо найти задание, которое он выполнит
наиболее эффективно, а транспортная компания заинтресована в оптимальном распределении
маршрутов между автомобилями.

Актуальность задачи
привела к тому, что для простой, двухиндексной постановки, то есть
к задаче, в которой участвуют $N$ работников и $N$ работ, были получены эффективные
алгоритмы решения, получающие точные решения за полиномиальное время. Последующие
прикладные задачи, а так же и внутренние потребности математики привели к
обобщению данной задачи, в частности был рассмотрена трехиндексная постановка.

К сожалению, Карпом было установлено, что подобная задача не решается за время
$\mathrm{O}(n^p)$, так как отностится к классу $\mathcal{NP}$-полных,
следовательно, для таких задач необходимо строить новые схемы
решения.
Очевидным решением является применение методов ветвей и границ, однако
данная схема ввиду вычислительной сложности не подходит для решения прикладных
задач, которые требуют быстрого ответа и доспускают некоторую неточность решения.
Тогда находят применение так называемые эвристические алгоритмы,
то есть алгоритм решения задачи, включающий практический метод, не являющийся
гарантированно точным или оптимальным, но достаточный для решения поставленной задачи.

Так как подобные класс схем очень разнообразен, и очень трудно изучить все
алгоритмы этого класса единовременно,
 актуальность работы состоит в изучении одного из приближенных методов,
 состоящем в уменьшении индексности, другими словами
 в уменьшении размерности задачи.

Целью работы является анализ одной из эвристических схем, предложенных
Гимади, с тем, чтобы установить, на сколько этот алгоритм может быть применен
на практике, при том, что автор гарантирует, что этот алгоритм сходится при
$n \rightarrow \infty $, что несколько больше $n$ в реальных задачах.
Для достижения поставленной цели необходимо решить следующие задачи:

\begin{itemize}
\item изучить математическую модель трехиндексной аксиальной задачи
\item провести анализ известных методов
\item изучить метод Гимади решения 3-АЗОН путём уменьшения индексности задачи
\item программно реализовать этот метод
\item провести анализ полученных результатов
\end{itemize}

\section{Трехиндексная акиальная задача о назначениях}
Введем понятие назначения. Мы можем
представлять назначение как некое биективное отображение $\phi$, которое ставит
элементы конечного множества $\mathrm{U}$ в соотвествие элементам конечного
множества $\mathrm{V}$. В тоже время назначение является перестановкой, которая записывается
в виде

\[
\left (
  \begin{tabular}{cccc}
  1 & 2 & \ldots & n\\
  $\varphi (1)$ & $\varphi (2)$ & \ldots & $\varphi (n)$
  \end{tabular}
\right )
\]

Каждой перестановке множества $\{1, 2, \ldots , n \}$ соответсвует единственная матрица
перестановок $\mathrm{X}_\varphi \in \mathrm{Matrix}_{n \times n}$, элементы котороый определяются как
\[
x_{ij} =
 \begin{cases}
   1 & \text{если } j = \varphi(i) \\
   0 & \text{иначе}
 \end{cases}
\]

Обозначим множество $\mathrm{S}_n$ как множество всех возможных перестановок множества
 $\{1, 2, \ldots , n \}$. Мощность этого множества $n!$.

\subsection{Линейная задача о назначениях}
Пусть дана матрица $n \times n$ весовых коэфициентов $C = (c_{ij})$.
Требуется минимизировать линейную форму

\[
  \sum^n_{i = 1} c_{i \varphi (i)}
\]

то есть линейная задача о назначениях может быть поставлена в виде

\[
  \min_{\varphi \in S_n} \sum^n_{i = 1} c_{i \varphi (i)}
\].

При этом, если перестановки задаются матрицей перестановок $\mathrm{X} = (x_{ij})$,
линейная задача о назначениях может быть записана как задача линейной оптимизации
\begin{align}
  & \min \displaystyle \sum^n_{i = 1} \displaystyle \sum^n_{j = 1} c_{ij} x_{ij} \\
  \text{ограничения:} & \displaystyle \sum^n_{i = 1} x_{ij} = 1 &(j = 1, \ldots, n) \\
  &\displaystyle \sum^n_{j = 1} x_{ij} = 1 &(i = 1, \ldots, n) \\
  & x_{ij} \in \{ 0, 1 \}
  &(i,j = 1, \ldots, n)
\end{align}

Ограничения  $(2) \-- (4)$ задают допустимое множество

В дальнейшем будем называть $\mathrm{X}$ матрицей назначений.


\textit{Наиболее частая постановка задачи}

Пусть требуется назначить $n$ работников на $n$ работ наиболее эфективным образом.
Предположим, что $j$ работнику требуется $c_{ij}$ времени, чтобы выполнить работу
$i$. Тогда нужно оптимизировать функцию
$\sum^n_{i = 1} c_{i \varphi(i)} $

\textit{Пример}

Рассмотрим матрицу назначений $C={c_{ij}}$
\[ \left(
\begin{matrix}
15 & 16 & 17 & 18 & 19 \\
1 & 1 & 3 & 1 & 16 \\
2 & 8 & 8 & 6 & 17 \\
2 & 2 & 2 & 8 & 18 \\
4 & 6 & 7 & 8 & 19
\end{matrix} \right)
\]

Строки, в терминах наиболее частой постановки задачи, соотвествуют работникам,
а столбцы -- работам.

Перестановка, обеспечивающая минимизацию линейной формы $\sum^n_{i = 1}
c_{i \varphi (i)}$ имеет

\[
\left (
  \begin{tabular}{ccccc}
  1 & 2 & 3 & 4 & 5\\
  1 & 5 & 2 & 4 & 3
  \end{tabular}
\right )
\]



или другими словами следует
\begin{itemize}
\item назаничить $1$ работника на $1$ работу
\item назаничить $2$ работника на $5$ работу
\item назаничить $3$ работника на $2$ работу
\item назаничить $4$ работника на $4$ работу
\item назаничить $5$ работника на $3$ работу
\end{itemize}


\section{Трехиндексная задача о назначениях}

Рассмотрим следующую задачу.
Составим расписание занятий, то есть нужно распределим $n$ занятий
среди $n$ свободных временных промежутков и среди $n$ свободных аудиторий.
Пусть $c_{ijk}$ соответсвует "стоимости" назначения курса $i$ на время $j$
в аудиторию $k$. Мы хотим найти такую перестановку $\varphi$, которая ставит
в соответствие курс к временному промежутку и такую перестановку $\psi$,
которая ставит
в соответствие курс к свободной аудитории, что "стоимость" назначения будет
минимальна. Эта задача -- пример трехиндексной аксиальной задачи о назначениях.
\[
 \min_{\varphi , \psi \in S_n} \sum^{n}_{i = 1} c_{i \varphi (i) \psi (i) }
\]

Также можно переписать эту задачу в виде

\begin{eqnarray*}
  & \min \displaystyle \sum^n_{i = 1} \displaystyle \sum^n_{j = 1} \displaystyle \sum^n_{k = 1}
  c_{ijk} x_{ijk} \\
  \text{s.t.}
  &\displaystyle \sum^n_{i = 1} \displaystyle \sum^n_{k = 1} x_{ijk} = 1  &(j = 1, \ldots, n) \\
  &\displaystyle \sum^n_{j = 1} \displaystyle \sum^n_{k = 1} x_{ijk} = 1  &(i = 1, \ldots, n) \\
  &\displaystyle \sum^n_{i = 1} \displaystyle \sum^n_{j = 1} x_{ijk} = 1  &(k = 1, \ldots, n) \\
  & x_{ijk} \in \{ 0, 1 \} &(i,j,k = 1, \ldots, n)
\end{eqnarray*}

Р. М. Карп показал, что эта задача является $\mathcal{NP}$ полной. В то время,
как линейная задача назначениях может быть решена быстрыми алгоритмами за полино
миальное время, точное решение для 3-АЗН может быть получено только алгоритмами полного перебора, которые
работают очень медленно (3-АЗН имеет $(n!)^2$ возможных решений), поэтому
применяют различные эвристические алгоритмы, которые дают за допустимое время
ответ, близкий к оптимальному.

\section{Обзор алгоритмов}

\subsection{О сложности}
Так же, Крама и Шпиксма исследовали теоретические
основы 3 АЗОН
в постановке на графах. Положим, что полный
3х дольный граф, вершины которого соответсвуют
индексам $i$, $j$, $k$. При этом, каждому ребру
$\left[ i, j \right]$ (или $\left[ j, k \right]$,
$\left[ i, k \right]$ соответсвенно )
ставится в соответсвие длина ребра $d \left( i, j
\right)$ ($d \left( j, k \right)$ , $d \left( i,k
\right)$ соответсвенно ), так же предположим,
что ребра удовлетворяют неравенству треугольника.
Имеем две модели определения весовых коэфициентов.

В первой определим весовой коэфициен $c_{ijk}$ как

$$
	c_{ijk} =
	d \left( i, j \right) +
	d \left( j, k \right) +
	d \left( i, k \right)
$$

Для второй модели положим $c_{ijk}$ как сумму двух
кратчайших длин ребер в треугольнике, образованным
вершинами $i$, $j$, $k$. Авторы показали, что для
каждой постановки соответсвующая 3-АЗОН является
НП полной, но при этом сконструировали
приближенне алгоритмы, которые получают ответы
не хуже чем $\frac{3}{2}$ от оптимального решения
в первом случае и не хуже чем $\frac{4}{3}$ для
второй постановки задачи.

Схожая модель была исследована Шпиксмой и
Воиджинжджером:
была рассмотрена плоскость с $3n$ точками на ней,
при этом $c_{ijk} = d(i,j) + d(i,k) + d(j,k)$, где
$d(i,j)$ - расстояние между точками $i,j$. Подобная
задача так же является НП полной.

Буркард, Рудольф, Воиджинжджер исследовали 3-АЗОН с
с разделяющимися коэфициентами вида $c_{ijk} = a_i b_j
d_j$,
где $a_i$,$ b_j $,$d_j$ - действительные неотрицательные
числа. Было показано. что в такой постановке задача
минимизации является НП полной, а задача максимизации
может быть решена за полиномиальное время.

Хансен и Кауфман построили алгоритм, схожий с венгерским
для класической ЗОН, сводящийся к поиску максимума в
гиперграфе.

\subsection{Эвристики}

Первым предложил примеменять эвристические алгоритмы
Пирскала. Наиболее часто применяются вариации точных
алгоритмов (так называемые метаэвристики). Мы можем
отметить жадный случайный адаптирующийся поиск,
разработанный
Аиэксом, Резенде и др., который показывает лучшие
результаты,
чем более ранние эвристики, а так же гибридный
генетический
алгоритм Хуанга и Лима.

Двумя группами ученых -- Крама, Банделт, Шпикма
и Буркард, Рудольф, Воиджинжджер -- разработали
эвристические
алгоритмы для 3-АЗОН с
с разделяющимися коэфициентами вида $c_{ijk} = a_i b_j
d_j$
и рассмотрели работу алгоритма в худшем случае.

\subsection{Асимптотическое поведение}

Вероятностное асимптотическое поведение аксиально 3-ЗОН
отличается от поведения классической задачи о назначениях.
В ряде работ, Грундель. Крокмал, Оливиера и др. изучили
поведение ожидаемого значения от наблюдаемой целевой функции.
Они показали, что это значение сходится к левой границе распределения 
весовых коэфициентов. В частности, пусть весовые коэфициенты, $c_{ijk}$ 
независимы и распределены нормально в $ \left[ 0 , 1 \right] $. 
Положим 
$$
 	z_{n} = \min \sum^n_{i=1} c_{i \varphi (i) \xi (i)}, 
$$
тогда 

$$
	\lim_{x\to\infty} \mathcal{M} (z_n) = 0
$$

\section{Алгоритм}
Через $\phi$ обозначим любую целочислено значащую функцию, при этом $1 < \phi_n < n$ 
\begin{enumerate}
\item Берем произвольную подстановку $\pi \in S_n$. Пусть $(d_{jk})$ - $n \times n$ 
матрица, содержащая элементы исходной матрицы $(c_{ijk})$, где индекс $j=\pi(i)$ такой, что
$$
d_{ij} = c_{\pi^{-1}(j)jk}
$$
для любых $1 \leq j$,$n \leq n$
Положим $f = 0 ; j =1 ; \mathrm{K}={1,2, \ldots , \phi_n}$. 
\item Выберем номер $\sigma(j)$ минимального элемента из множества $\mathrm{argmin} \, {d_{jk} | k \in K}$.
\item Полагаем $f = f + d_{j \sigma (j)} ; \mathrm{K} = \mathrm{K}  \setminus  {\sigma(j)} ; k=j+\phi_n$
\item Если $k \leq n $, то $K = K \bigcap {k}$.
\item $j = j + 1$
\item Повторяем п.2, пока j<n. В противном случае идем к п.7
\item Результатом работы алгоритма $\mathrm{A}(\phi_n)$ является значение функции $f$ целевой функции   
$f_{\mathrm{A}(\phi_n)}$. 
\end{enumerate}


\end{document}
