\documentclass[10pt,a4paper]{article}
\usepackage[utf8]{inputenc}
\usepackage[russian]{babel}
\usepackage[OT1]{fontenc}
\usepackage{amsmath, amsfonts, amssymb}
\author{Козловский Никита}
\title{Трехиндексные аксиальные задачи о назначениях}
\begin{document}
\section{abstract}
Задачи о назначениях часто встречаются в повседневной жизни: составление расписаний, 
распределение обязаностей между работниками и многие другие. Актуальность задачи
привела к тому, что для простой, двухиндексной постановки были получены эффективные 
алгоритмы решения. Однако при расширении задачи до трехиндексной было установлено, что
такая задача отностится к классу $\mathcal{NP}$ полных, поэтому для неё существует множество
алгоритмов. В этой работе мы рассмотрим некоторые из них и подробно исследуем алгоритм, предложенный
Э. Х. Гимади, который состоит в уменьшении размерности решаемой задачи.

\section{Основные понятия}
Для начала дадим классическое определение задачи о назначениях. 
Введем понятие назначения. Мы можем 
представлять назначение как некое биективное отображение $\phi$, которое ставит 
элементы конечного множества $\mathrm{U}$ в соотвествие элементам конечного 
множества $\mathrm{V}$. В тоже время назначение является перестановкой, которая записывается 
в виде 

\[ 
\left (
  \begin{tabular}{cccc}
  1 & 2 & \ldots & n\\
  $\varphi (1)$ & $\varphi (2)$ & \ldots & $\varphi (n)$
  \end{tabular}
\right )
\]

Каждой перестановке множества $\{1, 2, \ldots , n \}$ соответсвует единственная матрица 
перестановок $\mathrm{X}_\varphi \in \mathrm{Matrix}_{n \times n}$, элементы котороый определяются как 
\[
x_{ij} = 
 \begin{cases}
   1 & \text{если } j = \varphi(i) \\
   0 & \text{иначе}
 \end{cases}
\] 

Обозначим множество $\mathrm{S}_n$ как множество всех возможных перестановок множества 
 $\{1, 2, \ldots , n \}$. Мощность этого множества $n!$. 
 
\subsection{Линейная задача о назначениях}
Пусть дана матрица $n \times n$ весовых коэфициентов $C = (c_{ij})$

\end{document}